\documentclass{article}
\usepackage{graphicx} % Required for inserting images
\usepackage[
    backend=biber,
    style=apa,
  ]{biblatex}
\addbibresource{MRSinMRS.bib} %Import the bibliography file


\title{Methods provided by the automatic table tool based on MRSinMRS reporting standards}
\author{Antonia Susnjar, Antonia Kaiser, Gianna Nossa, Dunja Simicic, \\ Aaron Gudmundson}
\date{\today}

\begin{document}
\maketitle

\section{Methods}

\subsection{${}^{1}$H-MRS Spectroscopy}
\subsubsection{Acquisition}
All scanning procedures were conducted using a \textbf{Manufacturer} scanner equipped with a \textbf{coil info} coil operating at \textbf{B0} T. 
Before spectroscopy data acquisition, a \textbf{sequence for anatomical image} anatomical image (\textbf{info}) was obtained for accurate MRS voxel placement and subsequent segmentation. MRS data were acquired utilizing the \textbf{Protocol} sequence in combination with \textbf{WS method} water suppression method. The following acquisition parameters were used: repetition time/echo time (TR/TE) of \textbf{TR} s/\textbf{TE} ms, and \textbf{NumberOfAverages} transients recorded per acquisition of \textbf{total number of acquisitions} and a spectral width of \textbf{SpectralWidth} kHz. The volume of interest (VOI = \textbf{AnteriorPosteriorSize} x \textbf{LeftRightSize} x \textbf{CranioCaudalSize}) was placed in \textbf{VOILocation}. The static magnetic field homogeneity was adjusted using \textbf{shimming method} shimming method, the threshold for the acceptance of shim was set at \textbf{thresholds} using \textbf{reference peak} as the reference peak. 
For eddy current correction and quantification purposes, \textbf{number of averages}  transients of water-unsuppressed data were also acquired, using identical acquisition parameters.\\
-------------------------\\
*Note: 
!!! The positions in the text which could not be automatically filled using the table are highlighted in \textbf{BOLD}. Please fill in this information manually. 
!!! If you used multiple VOI or various VOI locations with the same protocol, please change the text accordingly to include VOI specific information (both location and threshold for shim).

\begin{itemize}
    \item IF Diffusion:
    !!! If you use a diffusion weighted sequence, please include additional information: diffusion time (\bigdelta), b-values, number of transients per b-value and the diffusion direction.
    \item IF editing:
    !!! If you use edited sequences, please add information: Precise frequency and bandwidth of the editing pulses for each sub experiment.
    \item IF two shot sequences were used:
    !!! Please specify if dynamic frequency correction was used. 
\end{itemize}

\subsubsection{Analysis}
The following preprocessing steps were applied using \textbf{known algorithm for preprocessing or in house developed algorithm} to the data before spectral analysis and quantification: \textbf{frequency and phase correction of individual transients}, \textbf{eddy-current correction}, \textbf{HSVD water removal}, \textbf{averaging}, \textbf{alignment and subtraction of sub spectra - for two shot sequences}, \textbf{RF coil combination}, \textbf{apodization}, \textbf{zero filling} were applied.
Quantitative analysis of the acquired spectroscopic data was performed using \textbf{quantification method}.
For linear combination modeling: The basis set used for linear-combination modeling was generated using \textbf{tool name} and it included \textbf{number} of metabolites: \textbf{all metabolite names}. \\
----------------------------\\
*Note:
\begin{itemize}
    \item IF tCr: All metabolite estimates were reported relative to total creatine (tCr = Cr + PCr) signal (metabolite-to-tCr ratio).
    \item IF Water: All metabolite estimates were reported as water-referenced pseudo-absolute molal units (moles of metabolite per kg of solute water). To account for tissue-related factors and control for cerebrospinal fluid (CSF) content within the voxel, \textbf{describe approach} was implemented.
\end{itemize}

!!! For more information about Preprocessing, analysis and quantification, please consult: Near, Jamie et al. “Preprocessing, analysis and quantification in single-voxel magnetic resonance spectroscopy: experts' consensus recommendations.” NMR in biomedicine vol. 34,5 (2021) \\ 
!!! if macromolecules (MM) were modeled by the chosen algorithm, please describe the MM modeling approach. If in vivo acquired macromolecules were used in the basis set, please add information about their acquisition and preparation (e.g. removal of metabolite residuals).


\subsection{Copyright Waiver}
The above text was automatically generated by the table tool based on MRSinMRS with the express intention that users should copy and paste this text into their manuscripts minimally changed.

\subsection{References Consensus:}
\nocite{*}
\printbibliography
\end{document}